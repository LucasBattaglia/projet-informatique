\documentclass[french, titlepage]{article}
\usepackage[utf8]{inputenc}
\usepackage[T1]{fontenc}
\usepackage[french]{babel}
\usepackage[colorlinks=true,linkcolor=black,urlcolor=blue]{hyperref}
\usepackage{geometry}

\geometry{hmargin=2.5cm,vmargin=2cm}

\title{\huge \textbf{Projet-Informatique}}
\author{BATTAGLIA Lucas | URBANOWSKI Théa \\ \\ MI3-Bin 08}
\date{2022}

\begin{document}


\maketitle

\tableofcontents

\newpage


\section{Partie Guidée}

    \subsection{Fonctions Majeures du Programme}
        \subsubsection{\ttfamily carte-to-chaine}
        \subsubsection{\ttfamily afficher-reussite}
        \subsubsection{\ttfamily init-pioche-fihier}
        \subsubsection{\ttfamily ecrire-fichier-reussite}
        \subsubsection{\ttfamily init-pioche-alea}
        \subsubsection{\ttfamily alliance(carte1, carte2)}
        \subsubsection{\ttfamily saut-si-possible}
        \subsubsection{\ttfamily une-etape-reussite}
        \subsubsection{\ttfamily reussite-mode-auto}
        \subsubsection{\ttfamily reussite-mode-manuel}
        \subsubsection{\ttfamily lance-reussite}


\section{Partie Extension}    
    
    \subsection{Interface Graphique Tkinter}
    
        Tkinter est une interface graphique dite de bureau, c'est-à-dire qu'il permet de créer des applications tel que Word, Excel ou des petits jeux comme le solitaire ou la réussite.\\
    
    Chaque objet de Tkinter (boutons, texte menu, etc...) est appelé WIDGET.\\
    A chaque fois l'on créer un WIDGET, il n'est pas afficher. On doit les afficher grâce à une des trois méthodes suivantes:\\
    
    \begin{itemize}
    
    \item .pack() : %underline 
    Place le WIDGET dans l'espace sans utiliser de coordonnées.\\
    On a la possibilité de placer notre WIDGET par l'ajout de l'argument size et par l'ajout d'une des valeurs suivantes.
    
    tkinter.CENTER  /  tkinter.BOTTOM  / tkinter.TOP  /  tkinter.LEFT  / tkinter.RIGHT
    
    Si, par la suite, on souhaite cacher le WIDGET, on utilisera la ligne de code : .pack$\_$forget()
    \\
    
	\item .grid() : %underline
    Place le WIDGET dans l'espace avec l'utilisation d'un semblant de coordonnées.\\
    On prend deux arguments, column (colonne) et row (ligne), où l'on placera notre WIDGET.\\
    Si la colonne et la ligne sont égales à 0, alors le widget sera dans le coin en haut à gauche.
    Le nombre de colonnes et de lignes n'est pas défini, à chaque rajout de ligne et/ou de colonne l'écran se divise.\\
    grid comprend aussi deux arguments supplémentaires mais optionnel : padx et pady, chaque correspondant à l'emplacement entre deux WIDGET, pour x et y.
    \\
    
    \item .place() : %underline
    Place le Widget dans l'espace en utilisant des coordonnées fixes et comprends deux arguments corrspondant aux coordonnées (x et y).\\
    On peut, en plus, autiliser l'argument optionnel anchor, représentation des pooints cardinaux : N, E, S, W, NE, NW, SE, SW.
    \\
    \end{itemize}
    
    \subsection{\ttfamily verifier-pioche}
    \subsection{Nombre moyen de tas}
        \subsubsection{\ttfamily res-multi-simulation}
        \subsubsection{\ttfamily statistiques-nb-tas}
    \subsection{Probabilité de gagner}
        \subsubsection{Graphique}
    \subsection{Interface graphique : Turtle}
    \subsection{Amélioration du mélange}
        \subsubsection{\ttfamily meilleur-echange-consecutif}
        \subsubsection{Modification plus approfondie}
        

\section{Bibliographie}

\end{document}