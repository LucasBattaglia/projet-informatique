\documentclass{article}
\usepackage[utf8]{inputenc}
\usepackage[T1]{fontenc}
\usepackage[french]{babel}
\usepackage[colorlinks=true,linkcolor=black,urlcolor=blue]{hyperref}
\usepackage{geometry}

\geometry{hmargin=2.5cm,vmargin=2cm}

\title{\huge \textbf{Projet-Informatique}}
\author{BATTAGLIA Lucas | URBANOWSKI Théa \\ \\ MI3-Bin 08}
\date{2022}

\begin{document}


\maketitle
\tableofcontents
\newpage


\section{Partie Guidée}

    \subsection{Fonctions Majeures du Programme}
        \subsubsection{\ttfamily carte-to-chaine}
        \subsubsection{\ttfamily afficher-reussite}
        \subsubsection{\ttfamily init-pioche-fihier}
        \subsubsection{\ttfamily ecrire-fichier-reussite}
        \subsubsection{\ttfamily init-pioche-alea}
        \subsubsection{\ttfamily alliance(carte1, carte2)}
        \subsubsection{\ttfamily saut-si-possible}
        \subsubsection{\ttfamily une-etape-reussite}
        \subsubsection{\ttfamily reussite-mode-auto}
        \subsubsection{\ttfamily reussite-mode-manuel}
        \subsubsection{\ttfamily lance-reussite}


\section{Partie Extension}    
    
    \subsection{Interface Graphique Tkinter}
    \subsection{\ttfamily verifier-pioche}
    \subsection{Nombre moyen de tas}
        \subsubsection{\ttfamily res-multi-simulation}
        \subsubsection{\ttfamily statistiques-nb-tas}
    \subsection{Probabilité de gagner}
        \subsubsection{Graphique}
    \subsection{Interface graphique : Turtle}
    \subsection{Amelioration du mélange}
        \subsubsection{\ttfamily meilleur-echange-consecutif}
        \subsubsection{Modification plus approfondie}
        

\section{Bibliographie}

\end{document}